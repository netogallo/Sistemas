%%%%%%%%%%%%%%%%%%%%%%%%%%%%%%%%%%%%%%%%%
% Programming/Coding Assignment
% LaTeX Template
%
% This template has been downloaded from:
% http://www.latextemplates.com
%
% Original author:
% Ted Pavlic (http://www.tedpavlic.com)
%
% Note:
% The \lipsum[#] commands throughout this template generate dummy text
% to fill the template out. These commands should all be removed when 
% writing assignment content.
%
% This template uses a Perl script as an example snippet of code, most other
% languages are also usable. Configure them in the "CODE INCLUSION 
% CONFIGURATION" section.
%
%%%%%%%%%%%%%%%%%%%%%%%%%%%%%%%%%%%%%%%%%

%----------------------------------------------------------------------------------------
%	PACKAGES AND OTHER DOCUMENT CONFIGURATIONS
%----------------------------------------------------------------------------------------

\documentclass{article}

\usepackage{fancyhdr} % Required for custom headers
\usepackage{lastpage} % Required to determine the last page for the footer
\usepackage{extramarks} % Required for headers and footers
\usepackage[usenames,dvipsnames]{color} % Required for custom colors
\usepackage{graphicx} % Required to insert images
\usepackage{listings} % Required for insertion of code
\usepackage{courier} % Required for the courier font
\usepackage{multirow}
\usepackage{hyperref}


% Margins
\topmargin=-0.45in
\evensidemargin=0in
\oddsidemargin=0in
\textwidth=6.5in
\textheight=9.0in
\headsep=0.25in

\linespread{1.1} % Line spacing

%----------------------------------------------------------------------------------------
%	CODE INCLUSION CONFIGURATION
%----------------------------------------------------------------------------------------

\definecolor{MyDarkGreen}{rgb}{0.0,0.4,0.0} % This is the color used for comments
\lstloadlanguages{c} % Load Perl syntax for listings, for a list of other languages supported see: ftp://ftp.tex.ac.uk/tex-archive/macros/latex/contrib/listings/listings.pdf
\lstset{language=[sharp]c, % Use Perl in this example
        frame=single, % Single frame around code
        basicstyle=\small\ttfamily, % Use small true type font
        keywordstyle=[1]\color{Blue}\bf, % Perl functions bold and blue
        keywordstyle=[2]\color{Purple}, % Perl function arguments purple
        keywordstyle=[3]\color{Blue}\underbar, % Custom functions underlined and blue
        identifierstyle=, % Nothing special about identifiers                                         
        commentstyle=\usefont{T1}{pcr}{m}{sl}\color{MyDarkGreen}\small, % Comments small dark green courier font
        stringstyle=\color{Purple}, % Strings are purple
        showstringspaces=false, % Don't put marks in string spaces
        tabsize=5, % 5 spaces per tab
        %
        % Put standard Perl functions not included in the default language here
        morekeywords={rand},
        %
        % Put Perl function parameters here
        morekeywords=[2]{on, off, interp},
        %
        % Put user defined functions here
        morekeywords=[3]{test},
       	%
        morecomment=[l][\color{Blue}]{...}, % Line continuation (...) like blue comment
        numbers=left, % Line numbers on left
        firstnumber=1, % Line numbers start with line 1
        numberstyle=\tiny\color{Blue}, % Line numbers are blue and small
        stepnumber=5 % Line numbers go in steps of 5
}

\newcommand{\horrule}[1]{\rule{\linewidth}{#1}}

% Creates a new command to include a perl script, the first parameter is the filename of the script (without .pl), the second parameter is the caption
\newcommand{\perlscript}[2]{
\begin{itemize}
\item[]\lstinputlisting[caption=#2,label=#1]{#1.cs}
\end{itemize}
}

\begin{document}

\begin{tabular}{l l}
\multirow{5}{*}{\includegraphics[width=2cm]{../Recursos/logo.png}} & Universidad del Istmo de Guatemala \\
 & Facultad de Ingenieria \\
 & Ing. en Sistemas \\
 & Dise\~no, Analisis y Fabricaci\'on de Sistemas \\
 & Prof. Ernesto Rodriguez - \href{mailto:erodriguez@unis.edu.gt}{erodriguez@unis.edu.gt} \\
\end{tabular}
\\\\\\

\begin{center}
        \horrule{0.5pt}
        \huge{Ex\'amen parcial \#1} \\
        \large{Tiempo de resoluci\'on: 90 minutos} \\
        \horrule{1pt}
\end{center}
\vspace{0.3cm}
Nombre: \noindent\rule{8cm}{0.4pt}
\\\vspace{0.1cm}\\\emph{Instrucciones: Responda las preguntas a continuaci\'on. Si necesita m\'as espacio, por favor solicitar
hojas adicionales al catedratico.}

% \perlscript{homework_example}{Sample Perl Script With Highlighting}

\section*{Seccion 1: Metodologias (50\%)}
\emph{Instrucciones: A continuaci\'on se muestra una descripci\'on de dos situaciones. Dadas las metodologias
        de desarrollo Waterfall y Scrum, escoga una metodologia para cada sistema, describa que ventajas
        y desventajas tendria utilizar esa metodologia en esa situaci\'on. {\bf Debe seleccionar una metodologia
        diferente para cada situacion.}}
\vspace{0.5cm}\\
\emph{Situaci\'on \#1: } Un grupo de aerolineas quiere ofrecerle a sus pasajeros la
mayor eficiencia para llegar a sus destinos. La causa principal de perdida de tiempo
para los pasajeros es perder una conexi\'on debido a un retraso y tener que
solicitar un nuevo vuelo con la aerolinea. Por eso, las aerolineas quieren crear una
aplicacci\'on movil y para los sistemas de entretenimiento del avion que le permita a
los pasajeros modificar su ruta mientras se encuentran volando. Debido a que hay varias 
aerolineas involucradas, y la coordinaci\'on de vuelos es un tema complicado y delicado, 
se desean probar varias alternativas diferentes para lograr que el sistema sea facil de
utilizar para los pasajeros, reduzca los costos de operaci\'on para las aerolinas y 
que el sistema sea robusto (un asiento solamente se ocupa por un pasajero).
\vspace{30cm}\\
\vspace{5cm}\\
\emph{Situaci\'on \#2: } Una peque\~na empresa que vende material de carpinteria tiene
un sistema de manejo de inventario que corre sobre la plataforma MS-DOS. Debido a que
la plataforma es vieja y el mantenimiento de la misma cada vez se torna m\'as dificil,
la empresa ha decidido contratar a una empresa para que migre el sistema a tecnologia
moderna. Hay muchas mejoras que podrian hacercele al sistema, pero la empresa por ser
peque\~na, no dispone de mucho presupuesto y sus empleados ya han usado el sistema
antiguo por a\~nos, por lo cual desean que el sistema sea lo m\'as similar posible al
anterior. Adicionalmente, es dificil comunicarse con el due\~no de la empresa ya que
a menudo se encuentra fuera del pueblo comprando material de carpinteria.
\vspace{20cm}\\

\section*{Secci\'on 2: L\'ogica de primer orden (50\%)}
\emph{Instrucciones: A continuaci\'on se presenta una l\'ogica de primer orden. La tarea
        consiste en convertir cada enunciado que se presenta a una expressi\'on de esta
        l\'ogica.}

\subsection*{Definicion de la l\'ogica}
El proposito de la siguiente l\'ogica es describir un sistema que almacena informaci\'on
relacionada a peliculas o series, similar a IMDB. A continuaci\'on se describen sus componentes.
\\\vspace{0.1cm}\\{\bf Simbolos constantes}
\begin{itemize}
        \item{Para toda pelicula o serie, se puede utilizar su titulo en letra arabica como
        simbolo para representarla. Ejemplo: ``Titanic'', ``El Rey Leon'', ``Rick and Morty'',
                etc.}
        \item{Para todo actor, se puede utilizar su nombre con letras arabicas como
        simbolo. Por ejemplo: ``Tom Hanks'',``Morgan Freeman'', etc.
        \item{Existen los siguientes simbolos para representar generos: ``Accion'',
                ``Suspenso'', ``Romance'' y ``Comedia'' }}
\end{itemize}
{\bf Simbolos predicados}
\begin{itemize}
        \item{Existe los predicados \texttt{es-genero},
                \texttt{es-actor} y \texttt{es-pelicula}, de aridad 1, para clasificar
                los diferentes simbolos.}
        \item{Existe el predicado \texttt{actua-en}, de aridad 2, para relacionar
                un actor con una pelicula.\\Ejemplo: \texttt{actua-en} ``Matrix'' ``Keano Reeves''}
        \item{Existe el predicado \texttt{tiene-genero}, de aridad 2, para relacionar
                una pelicula con un genero.\\Ej. \texttt{tiene-genero} ``Accion'' ``Matrix''}
        \item{Existe el predicado \texttt{actuan-juntos}, de cualquier aridad mayor que 1,
                el cual relaciona artistas que hayan actuado en la misma pelicula.\\Ejemplo:
                \texttt{actuan-juntos} ``Cameron Diaz'' ``Drew Barrymore'' ``Lucy Liu''}
\end{itemize}
{\bf Simbolos de funciones}
\begin{itemize}
        \item{Existe la funcion \emph{genero-principal}, de aridad 1, la cual recibe como
                parametro un actor y retorna el genero al cual pertenece la mayor cantidad
                de peliculas en que el actor haya participado.\\Ejemplo:
                \emph{genero-principal} ``Vin Diesel''} retornaria ``Accion''.
\end{itemize}

\subsection*{Expressiones}
\emph{Escribir una expressi\'on utilizando la l\'ogica de primer orden
presentada anteriormente que corresponda al texto que se presenta. La
primera expressi\'on se coloca como ejemplo.}
\\\vspace{0.1cm}\\{\bf Para toda pelicula, existe alg\'un genero.}\\
$\forall p\ \mathtt{es-pelicula}\ p\ \wedge\ (\exists\ g\ \mathtt{es-genero}\ g\wedge\ \mathtt{tiene-genero}\ g\ p)$
\vspace{20cm}
\\{\bf Toda pelicula debe pertenecer solamente a un genero.}
\\\vspace{3cm}
\\{\bf Si Johnny Depp, Orlando Bloom y Keira Knightley actuaron juntos,
        entonces existe al menos una pelicula donde hayan actuado ellos tres.}
\\\vspace{3cm}
\\{\bf Todo actor debe haber actuado en al menos una pelicula.}
\\\vspace{3cm}
\\{\bf Todo actor cuyo genero principal sea ``Suspenso'', debe haber
        actuado en al menos una pelicula de suspenso.}
\\\vspace{3cm}
\\{\bf Debe exister solamente un genero principal por cada actor.}
\\\vspace{3cm}
\\{\bf Si una pelicula tiene 3 actores, esos tres actores
        debieron haber actuado juntos.}

\section*{Extra (Hasta un 10\% adicional)}
\emph{Extienda la l\'ogica presentada anteriormente para poder expresar
los siguientes enunciados. Utilize un formato similar al formato
utilizado en este examen para extender dicha l\'ogica.}
\begin{itemize}
        \item{Toda pelicula debe tener una calificaci\'on.}
        \item{Los actores deben tener una calificaci\'on, la cual es
        el promedio de la calificaci\'on de las peliculas en que aparecen.}
        \item{Solamente existe una calificaci\'on por pelicula o actor.}
\end{itemize}

\end{document}
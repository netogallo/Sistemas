%%%%%%%%%%%%%%%%%%%%%%%%%%%%%%%%%%%%%%%%%
% Programming/Coding Assignment
% LaTeX Template
%
% This template has been downloaded from:
% http://www.latextemplates.com
%
% Original author:
% Ted Pavlic (http://www.tedpavlic.com)
%
% Note:
% The \lipsum[#] commands throughout this template generate dummy text
% to fill the template out. These commands should all be removed when 
% writing assignment content.
%
% This template uses a Perl script as an example snippet of code, most other
% languages are also usable. Configure them in the "CODE INCLUSION 
% CONFIGURATION" section.
%
%%%%%%%%%%%%%%%%%%%%%%%%%%%%%%%%%%%%%%%%%

%----------------------------------------------------------------------------------------
%	PACKAGES AND OTHER DOCUMENT CONFIGURATIONS
%----------------------------------------------------------------------------------------

\documentclass{article}

\usepackage{fancyhdr} % Required for custom headers
\usepackage{lastpage} % Required to determine the last page for the footer
\usepackage{extramarks} % Required for headers and footers
\usepackage[usenames,dvipsnames]{color} % Required for custom colors
\usepackage{graphicx} % Required to insert images
\usepackage{listings} % Required for insertion of code
\usepackage{courier} % Required for the courier font
\usepackage{multirow}
\usepackage{hyperref}
\usepackage[utf8]{inputenc}


% Margins
\topmargin=-0.45in
\evensidemargin=0in
\oddsidemargin=0in
\textwidth=6.5in
\textheight=9.0in
\headsep=0.25in

\linespread{1.1} % Line spacing

%----------------------------------------------------------------------------------------
%	CODE INCLUSION CONFIGURATION
%----------------------------------------------------------------------------------------

\definecolor{MyDarkGreen}{rgb}{0.0,0.4,0.0} % This is the color used for comments
\lstloadlanguages{c} % Load Perl syntax for listings, for a list of other languages supported see: ftp://ftp.tex.ac.uk/tex-archive/macros/latex/contrib/listings/listings.pdf
\lstset{language=[sharp]c, % Use Perl in this example
        frame=single, % Single frame around code
        basicstyle=\small\ttfamily, % Use small true type font
        keywordstyle=[1]\color{Blue}\bf, % Perl functions bold and blue
        keywordstyle=[2]\color{Purple}, % Perl function arguments purple
        keywordstyle=[3]\color{Blue}\underbar, % Custom functions underlined and blue
        identifierstyle=, % Nothing special about identifiers                                         
        commentstyle=\usefont{T1}{pcr}{m}{sl}\color{MyDarkGreen}\small, % Comments small dark green courier font
        stringstyle=\color{Purple}, % Strings are purple
        showstringspaces=false, % Don't put marks in string spaces
        tabsize=5, % 5 spaces per tab
        %
        % Put standard Perl functions not included in the default language here
        morekeywords={rand},
        %
        % Put Perl function parameters here
        morekeywords=[2]{on, off, interp},
        %
        % Put user defined functions here
        morekeywords=[3]{test},
       	%
        morecomment=[l][\color{Blue}]{...}, % Line continuation (...) like blue comment
        numbers=left, % Line numbers on left
        firstnumber=1, % Line numbers start with line 1
        numberstyle=\tiny\color{Blue}, % Line numbers are blue and small
        stepnumber=5 % Line numbers go in steps of 5
}

\newcommand{\horrule}[1]{\rule{\linewidth}{#1}}

% Creates a new command to include a perl script, the first parameter is the filename of the script (without .pl), the second parameter is the caption
\newcommand{\perlscript}[2]{
\begin{itemize}
\item[]\lstinputlisting[caption=#2,label=#1]{#1.cs}
\end{itemize}
}

\DeclareMathAlphabet{\mathpzc}{OT1}{pzc}{m}{it}

\begin{document}

\begin{tabular}{l l}
\multirow{5}{*}{\includegraphics[width=2cm]{../../Recursos/logo.png}} & Universidad del Istmo de Guatemala \\
 & Facultad de Ingenieria \\
 & Ing. en Sistemas \\
 & Analisis, dise\~no y fabricaci\'on de Sistemas \\
 & Prof. Ernesto Rodriguez - \href{mailto:erodriguez@unis.edu.gt}{erodriguez@unis.edu.gt} \\
\end{tabular}
\\\\\\

\begin{center}
        \horrule{0.5pt}
        \huge{Hoja de trabajo \#3/\#4} \\
        \large{Fecha de entrega: 11 de Mayo, 2018 - 11:59pm} \\
        Modalidad: Parejas o individual \\
        \horrule{1pt}
\end{center}

\emph{Instrucciones: Realizar cada uno de los ejercicios siguiendo sus respectivas
instrucciones. El trabajo debe ser entregado a traves de Github, en su repositorio del curso, colocado
en una carpeta llamada "Hoja de trabajo 3". Al menos que la pregunta indique diferente, todas las
respuestas a preguntas escritas deben presentarse en un documento formato pdf, el cual haya 
sido generado mediante Latex. Los ejercicios de programaci\'on deben ser colocados en una carpeta
llamada ``Programas", la cual debe colocarse dentro de la carpeta correspondiente a esta hoja de trabajo.}

% \perlscript{homework_example}{Sample Perl Script With Highlighting}

\section*{Recursos}
\begin{itemize}
        \item{\href{https://www.parity.io/}{Parity}: Un nodo de Ethereum.}
        \item{\href{https://solidity.readthedocs.io/en/v0.4.23/}{Solidity}: Documentaci\'on
        del lenguaje para \emph{smart contracts} Solidity.}
        \item{\href{http://faucet.ropsten.be:3001/}{Ethereum Ropsten Faucet}: Fuente de
        Ether (Ropsten) para el testnet. Permite obtener Ethereum (Ropsten) sin tener que
        minarlo.}
\end{itemize}

\section*{Descripci\'on}
El objetivo de esta tarea es desarrollar una applicaci\'on que utilize la tecnologia
de blockchains para funcionar. Dicha aplicaci\'on tendra dos componentes: (1) El
programa que correra sobre el blockchain \emph{Ethereum} y (2) una aplicaci\'on
cliente (desktop, mobil, web o consola) que permitira visualizar de forma
intuitiva el estado actual de la aplicaci\'on almacenado en el blockchain.

\section*{Descripci\'on}
Describir una aplicaci\'on que sera fabricada utilizando la tecnologia de
blockchains. En dicha descripci\'on debe incluir:
\begin{itemize}
        \item{Descripci\'on breve de la applicaci\'on}
        \item{Listado de los casos de uso de la applicaci\'on cliente}
        \item{Descripci\'on del estado que se almacenara en el blockchain}
        \item{Descripci\'on de las funciones que modificaran dicho estado}
        \item{Descripci\'on de los eventos que se generaran al modificar el estado.}
\end{itemize}

\section*{Aplicaci\'on de Blockchain}
Implementar mediante \emph{Smart Contracts} la logica de la aplicaci\'on utilizando
el lenguaje \emph{Solidity}. Se recomienda utilizar el cliente \emph{Parity} para
comunicarse con el blockchain de \emph{Ethereum}. Tambien se recomienda conectarse
al \emph{test net} (Ropsten) de \emph{Ethereum} ya que en el \emph{test net} no
tiene costo correr una aplicaci\'on. Para conectarse a Ropsten utilizando \emph{Parity}
se puede utilizar el comando:
\\\\
\texttt{> parity --chain ropsten}
\\\\
Media vez inicie \emph{Parity}, se puede interactuar con el mediante su interfaz web
acediendo al sitio \url{http://127.0.0.1:8180/}. Media vez acede a este sitio, puede
navegar a la seccion de ``Parity Wallet'' para crear una billetera de \emph{Ethereum}.
Luego de crear la billetera, puede acceder al sitio \href{http://faucet.ropsten.be:3001/}
para abonar su billetera con \emph{Ethereum}.
\\\\
La secci\'on de \emph{Contracts} de la billetera permite crear \emph{smart contracts} y
subirlos al blockchain de \emph{Ethereum}.
\\\\
Para aprender m\'as sobre \emph{Solidity} y como crear \emph{smart contracts}, puede visitar
los sitios:
\begin{itemize}
        \item{\url{https://www.ethereum.org/token}}
        \item{\url{https://medium.com/@bleev.in.tech/how-to-deploy-a-smart-contract-to-ethereum-testnet-e34fa5b10dd6}}
\end{itemize}
{\bf Nota: }A \emph{Parity} le tomara algunas horas sincronizarse con el blockchain de
\emph{Ethereum}.
\section*{Aplicac\'on Cliente}
T\'ambien se debe implementar una aplicaci\'on cliente que permita visualizar de forma intuitiva
la informaci\'on almacenada en el blockchain relacionada a la aplicaci\'on de la secci\'on anterior.
Puede utilizar la plataforma integrada de \emph{Parity} (\url{https://wiki.parity.io/Development-Overview})
para desarrollar su aplicaci\'on o un lenguaje externo (como Java) y conectarse al blockchain de \emph{Ethereum}
utilizando un adaptador como \href{https://github.com/ethereum/ethereumj}{Ethereumj}. La mayoria de lenguajes
populares de programaci\'on tienen adaptadores para conectarse al blockchain de \emph{Ethereum}.

\end{document}
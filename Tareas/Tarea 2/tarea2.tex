%%%%%%%%%%%%%%%%%%%%%%%%%%%%%%%%%%%%%%%%%
% Programming/Coding Assignment
% LaTeX Template
%
% This template has been downloaded from:
% http://www.latextemplates.com
%
% Original author:
% Ted Pavlic (http://www.tedpavlic.com)
%
% Note:
% The \lipsum[#] commands throughout this template generate dummy text
% to fill the template out. These commands should all be removed when 
% writing assignment content.
%
% This template uses a Perl script as an example snippet of code, most other
% languages are also usable. Configure them in the "CODE INCLUSION 
% CONFIGURATION" section.
%
%%%%%%%%%%%%%%%%%%%%%%%%%%%%%%%%%%%%%%%%%

%----------------------------------------------------------------------------------------
%	PACKAGES AND OTHER DOCUMENT CONFIGURATIONS
%----------------------------------------------------------------------------------------

\documentclass{article}

\usepackage{fancyhdr} % Required for custom headers
\usepackage{lastpage} % Required to determine the last page for the footer
\usepackage{extramarks} % Required for headers and footers
\usepackage[usenames,dvipsnames]{color} % Required for custom colors
\usepackage{graphicx} % Required to insert images
\usepackage{listings} % Required for insertion of code
\usepackage{courier} % Required for the courier font
\usepackage{multirow}
\usepackage{hyperref}
\usepackage[utf8]{inputenc}


% Margins
\topmargin=-0.45in
\evensidemargin=0in
\oddsidemargin=0in
\textwidth=6.5in
\textheight=9.0in
\headsep=0.25in

\linespread{1.1} % Line spacing

%----------------------------------------------------------------------------------------
%	CODE INCLUSION CONFIGURATION
%----------------------------------------------------------------------------------------

\definecolor{MyDarkGreen}{rgb}{0.0,0.4,0.0} % This is the color used for comments
\lstloadlanguages{c} % Load Perl syntax for listings, for a list of other languages supported see: ftp://ftp.tex.ac.uk/tex-archive/macros/latex/contrib/listings/listings.pdf
\lstset{language=[sharp]c, % Use Perl in this example
        frame=single, % Single frame around code
        basicstyle=\small\ttfamily, % Use small true type font
        keywordstyle=[1]\color{Blue}\bf, % Perl functions bold and blue
        keywordstyle=[2]\color{Purple}, % Perl function arguments purple
        keywordstyle=[3]\color{Blue}\underbar, % Custom functions underlined and blue
        identifierstyle=, % Nothing special about identifiers                                         
        commentstyle=\usefont{T1}{pcr}{m}{sl}\color{MyDarkGreen}\small, % Comments small dark green courier font
        stringstyle=\color{Purple}, % Strings are purple
        showstringspaces=false, % Don't put marks in string spaces
        tabsize=5, % 5 spaces per tab
        %
        % Put standard Perl functions not included in the default language here
        morekeywords={rand},
        %
        % Put Perl function parameters here
        morekeywords=[2]{on, off, interp},
        %
        % Put user defined functions here
        morekeywords=[3]{test},
       	%
        morecomment=[l][\color{Blue}]{...}, % Line continuation (...) like blue comment
        numbers=left, % Line numbers on left
        firstnumber=1, % Line numbers start with line 1
        numberstyle=\tiny\color{Blue}, % Line numbers are blue and small
        stepnumber=5 % Line numbers go in steps of 5
}

\newcommand{\horrule}[1]{\rule{\linewidth}{#1}}

% Creates a new command to include a perl script, the first parameter is the filename of the script (without .pl), the second parameter is the caption
\newcommand{\perlscript}[2]{
\begin{itemize}
\item[]\lstinputlisting[caption=#2,label=#1]{#1.cs}
\end{itemize}
}

\DeclareMathAlphabet{\mathpzc}{OT1}{pzc}{m}{it}

\begin{document}

\begin{tabular}{l l}
\multirow{5}{*}{\includegraphics[width=2cm]{../../Recursos/logo.png}} & Universidad del Istmo de Guatemala \\
 & Facultad de Ingenieria \\
 & Ing. en Sistemas \\
 & Analisis, dise\~no y fabricaci\'on de Sistemas \\
 & Prof. Ernesto Rodriguez - \href{mailto:erodriguez@unis.edu.gt}{erodriguez@unis.edu.gt} \\
\end{tabular}
\\\\\\

\begin{center}
        \horrule{0.5pt}
        \huge{Hoja de trabajo \#2} \\
        \large{Fecha de entrega: 4 de Marzo, 2018 - 11:59pm} \\
        \horrule{1pt}
\end{center}

\emph{Instrucciones: Realizar cada uno de los ejercicios siguiendo sus respectivas
instrucciones. El trabajo debe ser entregado a traves de Github, en su repositorio del curso, colocado
en una carpeta llamada "Hoja de trabajo 2". Al menos que la pregunta indique diferente, todas las
respuestas a preguntas escritas deben presentarse en un documento formato pdf, el cual haya 
sido generado mediante Latex. Los ejercicios de programaci\'on deben ser colocados en una carpeta
llamada ``Programas", la cual debe colocarse dentro de la carpeta correspondiente a esta hoja de trabajo.}

% \perlscript{homework_example}{Sample Perl Script With Highlighting}
"
\section*{Parte \#1: Codificaci\'on y Decodificaci\'on}
Para esta secci\'on, debe escribir dos funciones: ``Codificar''
y ``Decodificar''. Las funciones deben aceptar como parametro un
numero $n$ y un \texttt{string}. El numero corresponde a la base
que se esta utilizando y el string el mensaje que se intenta
codificar o decodificar. La funci\'on ``Codificar'' debe transformar
el \texttt{string} a un arreglo de numeros $n\in\mathbf{Z}^n$. De
la misma forma, la funci\'on ``Decodificar'' debe transformar el
arreglo de numeros al \texttt{string} original. Debe considerar que
el numero $n$ debe ser de precision arbitraria, de tal forma que utilizar
un \texttt{int} no da a basto.

\section*{Parte \#2: RSA}
Crear una funci\'on llamada ``RSAKeyGen'' que acepta como parametro
un numero $l$ que corresponde a la longitud de la llave publica.
Considerar que la llave privada debe ser aproximadamente de la misma
longitud. Es importante utilizar un RNG (Random Number Generator)
criptografico para que las llaves sean seguras. Esta funci\'on debe
retornar 3 valores: La llave publica $e$, la llave privada $d$ y el
modulo o base $n$ tal que permite encriptar y codificar mensajes
en el espacio $\mathbf{Z}^n$.

\section*{Parte \#3: Encriptar y Desencriptar}

Crear dos funciones: ``Encriptar'' y ``DesEncriptar''. La funci\'on
``Encriptar'' acepta una llave publica $e$, el modulo $n$ y un \texttt{string}.
Esta funci\'on debe producir un mensaje encriptado y codificado utilizando RSA.
De la misma forma, la funci\'on $DesEncriptar$ acepta la llave privada $d$,
el modulo $n$ y el mensaje ecriptado (un array de numeros) y recupera el \texttt{string}
original que fue encriptado.

\section*{Parte \#4: Hackear}

Crear una funci\'on llamada ``Hackear''. Esta funci\'on acepta una llave
publica $e$, un modulo $n$, un tiempo $t$ (en mili-segundos) y un mensaje encriptado (array de numeros) e intenta
recuperar el mensaje (\texttt{string}) original utilizando fuerza bruta. Si la funci\'on
no logra recuperar el mensaje en el tiempo dado en el parametro $t$, se debe
generar un error.

\section*{Parte \#5: Test de seguridad}
Escribir una prueba unitaria llamada ``TestSeguridadLineal''. Esta prueba unitaria
generara llaves RSA de longitud incremental empezando con una longitudo de 5. Luego debe encriptar un mensaje
e intentar vencer la ecriptaci\'on mediante la funci\'on ``Hackear''. El tiempo
limite para vencer la encriptaci\'on es la longitud de la llave publica en mili-segundos.
Hasta que longitud de llave fue posible vencer la encriptaci\'on?
\\\\
De la misma forma, escribir una prueba unitaria llamada ``TestSeguridadPolinomial''.
Esta prueba es igual a la prueba ``TestSeguridadLineal'' excepto que el tiempo limite
es la longitud $l$ de la llave publica elevado a la 3ra potencia ($l^3$). Hasta
que longitud de llave fue possible vencer la encriptaci\'on? Fue un incremento considerable
comparado con el ``TestSeguridadLineal''?

\end{document}
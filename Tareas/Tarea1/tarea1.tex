%%%%%%%%%%%%%%%%%%%%%%%%%%%%%%%%%%%%%%%%%
% Programming/Coding Assignment
% LaTeX Template
%
% This template has been downloaded from:
% http://www.latextemplates.com
%
% Original author:
% Ted Pavlic (http://www.tedpavlic.com)
%
% Note:
% The \lipsum[#] commands throughout this template generate dummy text
% to fill the template out. These commands should all be removed when 
% writing assignment content.
%
% This template uses a Perl script as an example snippet of code, most other
% languages are also usable. Configure them in the "CODE INCLUSION 
% CONFIGURATION" section.
%
%%%%%%%%%%%%%%%%%%%%%%%%%%%%%%%%%%%%%%%%%

%----------------------------------------------------------------------------------------
%	PACKAGES AND OTHER DOCUMENT CONFIGURATIONS
%----------------------------------------------------------------------------------------

\documentclass{article}

\usepackage{fancyhdr} % Required for custom headers
\usepackage{lastpage} % Required to determine the last page for the footer
\usepackage{extramarks} % Required for headers and footers
\usepackage[usenames,dvipsnames]{color} % Required for custom colors
\usepackage{graphicx} % Required to insert images
\usepackage{listings} % Required for insertion of code
\usepackage{courier} % Required for the courier font
\usepackage{multirow}
\usepackage{hyperref}
\usepackage[utf8]{inputenc}


% Margins
\topmargin=-0.45in
\evensidemargin=0in
\oddsidemargin=0in
\textwidth=6.5in
\textheight=9.0in
\headsep=0.25in

\linespread{1.1} % Line spacing

%----------------------------------------------------------------------------------------
%	CODE INCLUSION CONFIGURATION
%----------------------------------------------------------------------------------------

\definecolor{MyDarkGreen}{rgb}{0.0,0.4,0.0} % This is the color used for comments
\lstloadlanguages{c} % Load Perl syntax for listings, for a list of other languages supported see: ftp://ftp.tex.ac.uk/tex-archive/macros/latex/contrib/listings/listings.pdf
\lstset{language=[sharp]c, % Use Perl in this example
        frame=single, % Single frame around code
        basicstyle=\small\ttfamily, % Use small true type font
        keywordstyle=[1]\color{Blue}\bf, % Perl functions bold and blue
        keywordstyle=[2]\color{Purple}, % Perl function arguments purple
        keywordstyle=[3]\color{Blue}\underbar, % Custom functions underlined and blue
        identifierstyle=, % Nothing special about identifiers                                         
        commentstyle=\usefont{T1}{pcr}{m}{sl}\color{MyDarkGreen}\small, % Comments small dark green courier font
        stringstyle=\color{Purple}, % Strings are purple
        showstringspaces=false, % Don't put marks in string spaces
        tabsize=5, % 5 spaces per tab
        %
        % Put standard Perl functions not included in the default language here
        morekeywords={rand},
        %
        % Put Perl function parameters here
        morekeywords=[2]{on, off, interp},
        %
        % Put user defined functions here
        morekeywords=[3]{test},
       	%
        morecomment=[l][\color{Blue}]{...}, % Line continuation (...) like blue comment
        numbers=left, % Line numbers on left
        firstnumber=1, % Line numbers start with line 1
        numberstyle=\tiny\color{Blue}, % Line numbers are blue and small
        stepnumber=5 % Line numbers go in steps of 5
}

\newcommand{\horrule}[1]{\rule{\linewidth}{#1}}

% Creates a new command to include a perl script, the first parameter is the filename of the script (without .pl), the second parameter is the caption
\newcommand{\perlscript}[2]{
\begin{itemize}
\item[]\lstinputlisting[caption=#2,label=#1]{#1.cs}
\end{itemize}
}

\begin{document}

\begin{tabular}{l l}
\multirow{5}{*}{\includegraphics[width=2cm]{../../Recursos/logo.png}} & Universidad del Istmo de Guatemala \\
 & Facultad de Ingenieria \\
 & Ing. en Sistemas \\
 & Analisis, dise\~no y fabricaci\'on de Sistemas \\
 & Prof. Ernesto Rodriguez - \href{mailto:erodriguez@unis.edu.gt}{erodriguez@unis.edu.gt} \\
\end{tabular}
\\\\\\

\begin{center}
        \horrule{0.5pt}
        \huge{Hoja de trabajo \#1} \\
        \large{Fecha de entrega: 7 de Febrero, 2018 - 11:59pm} \\
        \horrule{1pt}
\end{center}

\emph{Instrucciones: Realizar cada uno de los ejercicios siguiendo sus respectivas
instrucciones. El trabajo debe ser entregado a traves de Github, en su repositorio del curso, colocado
en una carpeta llamada "Hoja de trabajo 1". Al menos que la pregunta indique diferente, todas las
respuestas a preguntas escritas deben presentarse en un documento formato pdf, el cual haya 
sido generado mediante Latex. Los ejercicios de programaci\'on deben ser colocados en una carpeta
llamada ``Programas", la cual debe colocarse dentro de la carpeta correspondiente a esta hoja de trabajo.}

% \perlscript{homework_example}{Sample Perl Script With Highlighting}

\section*{Contexto}

Usted es contratado por una empresa llamada ``Musicon", la cual quiere dise\~nar un sistema
de nueva generaci\'on para escuchar musica. Debido a que ya existen varios sistemas capaces
``adivinar'' que musica le gustaria a un usuario, la empresa quiere llevar el sistema al
siguiente nivel de tal forma que pueda utilizarse cuando hay varias personas escuchando
musica al mismo tiempo y sea capaz de ``adivinar'' que canciones le gustaria a todas
las personas que estan escuchando actualmente. Adicionalmente, la empresa quiere aprovechar
el internet of things, y vender hardware que puedan mejorar el rendimiento del sistema.

\section*{Ejercicio \#1: Dise\~no (20\%)}
En base a la descripci\'on anterior, elabore una especificaci\'on detallada y concreta
que defina como se construira el sistema mencionado anteriormente. Describa los mecanismos
que se utilizaran para cumplir con los requisistos de dicho sistema. Si lo considera
necesario, tenga la libertad de discutir con el cliente (Ernesto Rodriguez) sus ideas
en caso que algo no este claro o quisiera recibir retro-alimentaci\'on. Trate que su
descripci\'on sea completa, pero al mismo tiempo consica y ordenada, de manera que sea
facil entender (a un grado alto de abstracci\'on) como funcionara el sistema.

No es necesario que escriba codigo o elabore ning\'un diagrama. Sin embargo, puede apoyarse
de estas herramientas (y otras) si usted considera que estas ayudarian a describir el
sistema.

\section*{Ejercicio \#2: Metodologias (20\%)}
Supongamos que usted trabaja para la empresa que desarollara el sistema ``Musicon''.
El analista lider del proyecto ha decidido utilizar la metodologia ``Waterfall'' para
construir el sistema. ¿Usted apoyaria esa desici\'on? ¿Por que? En caso contrario,
¿que alternativa utilizaria? y ¿Que ventajas tiene dicha alternativa?

\section*{Ejercicio \#3: Definici\'on de L\'ogica (20\%)}
Defina una l\'ogica de primer orden\cite{FLL} que le permita formalizar propiedades y aspectos
del sistema ``Musicon''. Como recordatorio, esto consiste en definir:
\begin{itemize}
        \item Los simbolos constantes.
        \item Los simbolos de predicados
        \item Los simbolos de funciones
\end{itemize}
Para cada simbolo que defina, debe dar una breve descripcion del significado del simbolo
y una breve justificaci\'on de su importancia para esta l\'ogica. Tambien debe especificar
la \emph{aridad} de los simbolos que la requieran.

\section*{Ejercicio \#4: Propiedades l\'ogicas (20\%)}
Formalize al menos 5 aspectos o propiedades del sistema ``Musicon'' mediante expresiones
de la l\'ogica de primer orden que dise\~o en la secci\'on anterior. Por cada,
expresi\'on, tambien debe dar una breve descripci\'on en lenguaje natural
del significado de la expresi\'on.

\section*{Ejercicio \#5: Interpretaci\'on (20\%)}
Describa el dominio $\mathcal{A}$ de la logica definida anteriormente. Para ello debe:
\begin{itemize}
        \item{Describir el conjunto $\mathit{A}$ de objetos contenidos en $\mathcal{A}$}
        \item{Describir las los conjuntos abarcados por cada predicado de la logica}
        \item{Describir las funciones mediante conjuntos de tuplas de la logica.}
\end{itemize}

\bibliography{../../Referencias/referencias}
\bibliographystyle{plain}

\end{document}
\documentclass{beamer}
\usepackage[utf8]{inputenc}
\usepackage{hyperref}
\usepackage{multicol}
\usepackage{hyperref}
\usepackage{graphicx}
\usepackage{booktabs}
\usepackage[font={small,it},labelfont=bf]{caption}
\hypersetup{
    colorlinks=true,
    urlcolor={blue!40!black},
    linkcolor={red!50!black}
}

\inputencoding{utf8}

\mode<presentation> {
    \usetheme{Madrid}
}

\title[Ciclo de vida]{Ciclo de vida de un sistema}
\author{Prof. Ernesto Rodriguez}
\institute{
    Universidad del Itsmo \\
    \medskip \textit{erodriguez@unis.edu.gt}
}

\date[\today]{}

\begin{document}

\begin{frame}
\titlepage
\end{frame}

\begin{frame}
    \frametitle{Sistema de informaci\'on}
    \begin{itemize}
        \item{Responsable de recolectar, almacenar, procesar y presentar informaci\'on.}
        \item{Es utilizado por varios \emph{actores} de diferentes campos.}
        \item{Implementa un conjunto de \emph{reglas}, las cuales definen su operaci\'on.}
        \item{Puede estar compuesto de diferentes sub-sistemas.}
        \item{Debe garantizar que se cumplan alg\'uans \emph{invariantes} en todo momento.}
        \item{\emph{invariante}: Propiedad que tienen un conjunto de objetos, que se preserva aun cuando un objeto es transformado. Ej. $\forall a . a \in \mathtt{Array}:\mathtt{a.length} \geq 0$ }
    \end{itemize}

\end{frame}

\begin{frame}
    \frametitle{Ejemplo: Un Banco en linea}
    \begin{itemize}
        \item{¿Que información se esta procesando?}
        \item{¿Como se debe almacenar dicha informaci\'on?}
        \item{¿Como se debe procesar dicha informaci\'on?}
        \item{¿Que actores utilizaran el sistema?}
        \item{¿Que servicios necesita nuestro sistema para funcionar correctamente?}
        \item{¿Que condiciones extraordinarias puede enfrentar nuestro sistema durante su uso?}
        \item{¿Que invariantes existen en el sistema?}
    \end{itemize}
\end{frame}

\begin{frame}
    \frametitle{Retos al desarrollar un Sistema}
    \begin{itemize}
        \item{El sistema debe acoplarse a personas de diferentes campos, culturas y necesidades.}
        \item{El sistema debe ser desarrollado por muchas personas, tambien de diferentes campos.}
        \item{El sistema debe garantizar cumplir con muchos requisitso, posiblemente conflictivos.}
        \item{El sistema debe ser completado antes de una fecha.}
        \item{El sistema debe ser mantenido y respaldado luego de ser entregado.}
    \end{itemize}
\end{frame}

\begin{frame}
\frametitle{Etapas de desarrollo}
\begin{itemize}
    \item Recolecci\'on de requisitos
    \item Investigaci\'on del problema y requisitos ¿Ya existen soluciones?
    \item Administraci\'on del equipo
    \item Dise\~no del sistema
    \item Programaci\'on
    \item Evaluar el sistema
    \item Documentaci\'on del sistema
    \item Soporte y mantenimiento
\end{itemize}
\end{frame}

\begin{frame}
\frametitle{Retos durante el desarrollo de un Sistema}

\begin{tabular}{p{8cm} p{3cm}}
    \begin{itemize}
        \item{Los requisitos del cliente pueden cambiar.}
        \item{El equipo de desarrolladores puede cambiar $\Rightarrow$ \emph{Bus factor:}
            La probabilidad que el desarrollo del sistema fracase si uno de los desarrolladores
            es arroyado por un bus.}
        \item{Alg\'un requisito del sistema resulta ser
            imposible $\Rightarrow$ ¿Retirar el requisito
            o modificarlo?}            
        \item{El ambiente de desarrollo difiere drasticamente
            del ambiente de producci\'on: Firewalls, proxies,
            arquitecturas, etc.}
        \item{En general: {\bf Todo es incierto, los datos son
            poco confiables y las personas de competencia dudosa...}}
    \end{itemize} &
    \raisebox{-\totalheight}{
        \includegraphics[width=3cm]{parkbird.png}
    }\captionof{figure}{Obtained from XKCD \cite{XkcdBirdPark}} \\
\end{tabular}
\end{frame}

\begin{frame}
    \frametitle{¿Como ordenar las fases de desarrollo?}
    \begin{itemize}
    \item ¿Como manejar la incertidumbre y falta de informaci\'on?
    \item ¿Quienes son los accionistas del sistema? $\Rightarrow$ a menudo no son el cliente.
    \item ¿Como recibir retro-alimentaci\'on del cliente?
    \item ¿Como manejar cambios de requisitos?
    \end{itemize}
\end{frame}

\begin{frame}
\frametitle{Referencias}
\bibliography{../../Referencias/referencias}
\bibliographystyle{plain}
\end{frame}

\end{document}
\documentclass{beamer}
\usepackage[utf8]{inputenc}
\usepackage{hyperref}
\usepackage{multicol}
\usepackage{hyperref}
\usepackage{graphicx}
\usepackage{booktabs}
\usepackage[font={small,it},labelfont=bf]{caption}
\hypersetup{
    colorlinks=true,
    urlcolor={blue!40!black},
    linkcolor={red!50!black}
}

\inputencoding{utf8}

\mode<presentation> {
    \usetheme{Madrid}
}

\title{Introducci\'on a Logica formal}
\author{Prof. Ernesto Rodriguez}
\institute{
    Universidad del Itsmo \\
    \medskip \textit{erodriguez@unis.edu.gt}
}

\date[\today]{}

\begin{document}

\begin{frame}
\titlepage
\end{frame}

\begin{frame}
    \frametitle{La Situaci\'on:}
    \begin{itemize}
        \item{Un sistema tiene una lista de requisitos.}
        \item{El sistema debe cumplir con reglas, estas se pueden expresar como \emph{invariantes}.}
        \item{Cada vez que el sistema cambia su estado, se debe asegurar que las invariantes se cumplan.}
        \item{{\bf Ejemplos:}
            \begin{itemize}
                \item{Un usuario transfiere fondos a otro usuario.
                    \begin{itemize}
                        \item{El saldo de todos los usuarios es mayor o igual a cero.}
                        \item{El saldo de todos los usuarios es la suma de las transacciones.}
                    \end{itemize}
                \item{Un tren debe ser colocado en el riel que lo lleve a su destino.
                    \begin{itemize}
                        \item{Jamas deben haber dos trenes en la misma posici\'on.}
                        \item{El tren debe alcanzar su destino en tiempo finito.}
                    \end{itemize}
                }
                }
            \end{itemize}
        }
    \end{itemize}

\end{frame}

\begin{frame}
    \frametitle{La Situaci\'on:}
    \begin{itemize}
        \item{El sistema puede tener cientos de requisitos e invariantes.}
        \item{En el sistema trabajan muchas personas.}
        \item{Es altamente costoso verificar que los requisitos e
            invariantes se c\'umplen cada vez que se actualiza el codigo.}
        \item{El sistema esta distribuido.}
        \item{El sistema evoluciona constantemente.}
        \item{El lenguaje natural es ambig\"uo}
    \end{itemize}
\end{frame}

\begin{frame}
    \frametitle{Incluso si el sistema esta bien definido:}
    \begin{itemize}
        \item{No es possible correr el sistema en todos
            los casos posibles (no habria necesidad del sistema)}
        \item{El \"halting problem\" no es decidible.}
        \item{Las propiedades de comportamiento de un programa no
            son decidibles: Teorema de Rice.}
        \item{No importa que metodo utilizemos para validar el sistema,
            el m\'etodo debe hacerlo en tiempo finito.}
    \end{itemize}
    {\bf ¿Podemos hacer algo?}
\end{frame}

\begin{frame}
    \begin{center}
    \includegraphics[width=6cm]{yes.jpg}
    \end{center}
\end{frame}

\begin{frame}
    \frametitle{Verificaci\'on automatizada}
    \begin{itemize}
        \item{Se ejecutan continuamente.}
        \item{Existen varios metodos:
            \begin{itemize}
                \item{Pruebas:
                    \begin{itemize}
                        \item{Casos individuales}
                        \item{Casos aleatoreos}
                        \item{Casos construidos inductivamente}
                    \end{itemize}
                }
                \item{Verificaci\'on formal:
                    \begin{itemize}
                        \item{Analisis de flujo}
                        \item{Theorem provers: Z3, Coq}
                        \item{Lenguajes de modelaci\'on (Promela)}
                    \end{itemize}
                }
                \item{Sistemas de tipos:
                    \begin{itemize}
                        \item{Sistemas de primer orden}
                        \item{Sistemas de orden superior}
                        \item{Sistemas de categorias superiores (Higher kinds)}
                    \end{itemize}
                }
            \end{itemize}
        \item{Pueden validar numerosas condiciones en poco tiempo.}
        }
    \end{itemize}
\end{frame}

\begin{frame}
    \frametitle{M\'etodos formales vs. Pruebas automatizadas}
    \begin{tabular}{|p{5cm}|p{5cm}|}
    \hline
    {\bf Metodos formales} & {\bf Pruebas automatizadas} \\
    \hline
    Verifican todos los casos & Verifican numero finito de casos \\
    \hline
    No decidibles & Decidibles \\
    \hline
    Dificiles de implementar & Faciles de implementar \\
    \hline
    Alto costo a cambios en el sistema & Mediano costo a cambios en el sistema \\
    \hline
    \end{tabular}
\\La base de todo metod de verificaci\'on es: {\bf Logica formal}
\end{frame}

\begin{frame}
    \frametitle{Logica Formal: Ejemplos}
    \begin{itemize}
        \item{Lenguaje formal para describir objetos}
        \item{Consiste de:
            \begin{itemize}
                \item{Predicados}
                \item{Igualdad}
                \item{Negaci\'on}
                \item{Operadores boleanos}
                \item{Cuantificadores}
            \end{itemize}
        }
        \item{Ejemplos:
            \begin{itemize}
                \item{Representaci\'on de cuentas}
                \item{Representaci\'on de un banco}
                \item{Abonar cuenta}
            \end{itemize}
        }
        
    \end{itemize}

\end{frame}

\begin{frame}
    \frametitle{Sematicas operacionales}
    \begin{itemize}
        \item{Se utilizan para describir el comportamiento de un sistema.}
        \item{Describen como se transforma el estado de un sistema.}
        \item{Ejemplos:
            \begin{itemize}
                \item{Transferir fondos}
                \item{Abonar cuenta}
                \item{Retirar fondos}
            \end{itemize}
        }
    \end{itemize}
\end{frame}

\end{document}
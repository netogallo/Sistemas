\documentclass{beamer}
\usepackage[utf8]{inputenc}
\usepackage{hyperref}
\usepackage{multicol}
\usepackage{hyperref}
\usepackage{graphicx}
\usepackage{booktabs}
\usepackage[font={small,it},labelfont=bf]{caption}
\hypersetup{
    colorlinks=true,
    urlcolor={blue!40!black},
    linkcolor={red!50!black}
}

\inputencoding{utf8}

\mode<presentation> {
    \usetheme{Madrid}
}

\title{Logica de primer orden}
\author{Prof. Ernesto Rodriguez}
\institute{
    Universidad del Itsmo \\
    \medskip \textit{erodriguez@unis.edu.gt}
}

\date[\today]{}

\begin{document}

\begin{frame}
\titlepage
\end{frame}

\begin{frame}
    \frametitle{Motivaci\'on}
    \begin{itemize}
        \item{La logica es una de las herramientas m\'as universales que hay
        para describir objetos formalmente. En nuestro caso, sistemas.}
        \item{Existen infinitas l\'ogicas diferentes. Ejemplos:
            \begin{itemize}
                \item{L\'ogica de primer orden}
                \item{L\'ogica de segundo orden}
                \item{L\'ogica del $400^{avo}$ orden}
                \item{Logica temporal lineal}
                \item{Logica de Hoare}
            \end{itemize}
        }
        \item{La l\'ogica nos permite traducir propiedades y hechos del
        mundo real a un lenguaje que entienden las computadoras.}
    \end{itemize}
\end{frame}

\begin{frame}
    \frametitle{{Motivaci\'on}}
    \begin{itemize}
        \item{La diferencia principal entre las logicas es su \emph{expresividad}
        (m\'as sobre eso despues).}
        \item{La \emph{Logica de primer orden} ofrece un buen balance
            entre expresividad, practicalidad y simplicidad.}
        \item{La l\'ogica de primer orden es una herramienta muy util en la inteligencia
            artificial. Sistemas expertos y bases de conocimiento como Watson y Cync son
            impulsadas por alg\'una variante de esta logica.}
    \end{itemize}
\end{frame}

\begin{frame}
    \frametitle{Ejemplo:}
    \begin{center}
        $\forall h.\mathtt{Humano}\ h \Rightarrow\ \mathtt{Mortal}\ h$
    \end{center}
    \begin{center}
        $\forall x\forall y.(\mathtt{Entero}\ x\wedge\mathtt{Entero}\ y\wedge(\exists n.x=n+n)\wedge(\exists m.y=m+m))\Rightarrow \exists q.x+y=q+q$
    \end{center}
\end{frame}

\begin{frame}
    \frametitle{Vocabulario}
    \begin{itemize}
        \item{La logica de primer orden en realidad es una familia de logicas.}
        \item{El vocabulario de la logica de primer orden viene de varios dominios. Ejemplo:
            \begin{itemize}
                \item{Si uno intenta describir numeros naturales, simbolos como: $+$, $-$, $0$, ect.}
                \item{Si uno intenta describir un sistema experto medico: \texttt{diagnostico}, \texttt{tratamiento}, ect.}
            \end{itemize}
        }
    \end{itemize}
\end{frame}

\begin{frame}
    \frametitle{{Vocabulario}}
    \begin{itemize}
        \item{Los simbolos que comparten \emph{todas} las logicas de primer orden son:
        \begin{itemize}
            \item{{\bf Variables:} $x_1$, $x_2$, $y$, $z$, $\ldots$}
            \item{{\bf Conectivos l\'ogicos:} $\wedge$, $\vee$, $\neg$, $\Rightarrow$, $\Leftrightarrow$}
            \item{{\bf Cuantificadores:} $\forall$ y $\exists$}
            \item{{\bf La igualdad:} $=$}
            \item{{\bf parentesis:} $(x)$}
        \end{itemize}
        }
        \item{Los simbolos especificos de un domino se definen as:
        \begin{itemize}
            \item{Por cada numero $n$, un conjunto de simbolos de predicados con aridad $n$. Ej. $\mathtt{Mortal}\ x$}
            \item{Por cada numero $n$, un conjunto de simbolos de funciones con aridad $n$. Ej. $distancia\ x\ y$}
            \item{Un conjunto de simbolos constantes. Ej. $0$, $1$}
        \end{itemize}
        }
    \end{itemize}
\end{frame}

\begin{frame}
    \frametitle{Observaciones}
    \begin{itemize}
        \item{El conjunto de simbolos en una logica de primer orden puede ser
        infinito, incluso de cardinalidad arbitraria! Ie. todos los numeros
        reales podrian ser un simbolo.}
        \item{Los simbolos de la logica de primer orden se dividen en tres tipos:
        funciones, predicados y constantes.}
        \item{Los simbolos de la l\'ogica de primer orden tienen una propiedad
        llamada \emph{aridad}.}
        \item{Los simbolos constantes denotan objetos especificos al dominio.}
        \item{Los predicados de aridad 1 denotan propiedades de objetos.}
        \item{Los predicados de aridad mayor a 1 denotan relaciones entre objetos.}
    \end{itemize}
\end{frame}

\begin{frame}
    \frametitle{Sintaxis: Terminos}
    Un \emph{Termino} en logica de primer orden se define recursivamente com:
    \begin{itemize}
        \item{Un simbolo constante es un \emph{Termino}}
        \item{Una variable es un termino.}
        \item{Si $t_1\ldots t_n$ son terminos y $f$ es una funci\'on de aridad $n$, $f\ t_1\ldots t_n$ es un termino.}
    \end{itemize}
    Ejemplos, dado el conjunto de simbolos $S:=\{\mathtt{Alice},\mathtt{Bob}\}$ y la funcion $padre\_de$:
    \begin{itemize}
        \item{$\mathtt{Alice}$}
        \item{$padre\_de\ \mathtt{Alice}$}
        \item{$padre\_de\ padre\_de\ \mathtt{Bob}$}
    \end{itemize}
\end{frame}

\begin{frame}
    \frametitle{Sintaxis: Expressiones}
    Las expresiones de definen como:
    \begin{itemize}
        \item{Dados los terminos $t_0$ y $t_1$, $t_0=t_1$ es una expression.}
        \item{Dados los terminos $t_0\ldots t_n$ y un predicado $P$ de aridad $n$, $P\ t_0\ldots t_n$ es una expression.}
        \item{Dada la expresion $\psi$, $\neg\psi$ es una expresi\'on.}
        \item{Dadas las expresiones $\psi_1$ y $\psi_2$, ($\psi_1\wedge\psi_2$), ($\psi_1\vee\psi_2$), ($\psi_1\Rightarrow\psi_2$) y ($\psi_1\Leftrightarrow\psi_2$) son expressiones.}
        \item{Dada la expresi\'on $\psi$ y una variable $x$, ($\exists x\ \psi$) y ($\forall x\ \psi$) son expresiones.}
    \end{itemize}
    Ejempos:
    \begin{itemize}
        \item $\exists x\ padre\_de\ \mathtt{Alice}=x$
        \item $\forall y\ \exists x\ padre\_de\ y=x\wedge \neg(x=y)$
    \end{itemize}
\end{frame}

\begin{frame}
    \frametitle{Expresividad}
    \begin{itemize}
        \item{Un objeto, sistema, ect. puede tener infinitas propiedades y pueden
        haber infinitos hechos respecto a el.}
        \item{Una l\'ogica (sea la que sea) no puede expresar todas las propiedades
        de un objeto (a pesar de poder expresar infinitas propiedades y enunciados).}
        \item{La medida de la cantidad de propiedades que puede expresar una l\'ogica
        se llama \emph{expresividad}.}
        \item{La l\'ogica de primer orden permite expresar todas las propiedades
        \emph{parcialmente computables}.}
        \item{Algunas logicas, como la logica de segundo orden tienen mayor expresividad.
        Ejemplo, principio de inducci\'on: $\forall P\ ((P\ 0)\wedge(\forall n\ P\ n\Rightarrow P\ (n+1)))\Rightarrow \forall n\ P\ n$}
        \item{Las propiedades expresables por la l\'ogica de segndo orden no son computables.}
    \end{itemize}
\end{frame}

\end{document}
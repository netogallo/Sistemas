\documentclass{beamer}
\usepackage[utf8]{inputenc}
\usepackage{hyperref}
\usepackage{multicol}
\usepackage{hyperref}
\usepackage{graphicx}
\usepackage{booktabs}
\usepackage[font={small,it},labelfont=bf]{caption}

\DeclareMathAlphabet{\mathpzc}{OT1}{pzc}{m}{it}

\hypersetup{
    colorlinks=true,
    urlcolor={blue!40!black},
    linkcolor={red!50!black}
}

\inputencoding{utf8}

\mode<presentation> {
    \usetheme{Madrid}
}

\title{Logica de primer orden: Calculo}
\author{Prof. Ernesto Rodriguez}
\institute{
    Universidad del Itsmo \\
    \medskip \textit{erodriguez@unis.edu.gt}
}

\date[\today]{}

\begin{document}

\begin{frame}
\titlepage
\end{frame}

\begin{frame}
    \frametitle{Motivaci\'on}
    \begin{itemize}
        \item{Solamente hemos estudiado que significa que una expresi\'on es cierta.}
        \item{No tenemos herramientas para encotrar una demostraci\'on que dicha expresi\'on es cierta.}
        \item{Existen varias de encontrar demostraciones.}
        \item{En esta clase estudiaremos un \emph{calculo sequencial} para ello.}
        \item{La belleza de la l\'ogica no solamente es que podemos determinar
            si algo es verdad o no, sino que tambien podemos encontrar una demostraci\'on
            (si existe).}
        \item{El calculo que se presentara en esta clase es un \emph{calculo completo}}
    \end{itemize}
\end{frame}

\begin{frame}
    \frametitle{Calculo sequencial}
    {\bf Ejemplo: } Demostraci\'on por contradicci\'on.
    \begin{tabular}{p{7cm} p{3cm}}
        \begin{enumerate}
            \item{Queremos demostrar que $\varphi$ es verdad.}
            \item{Assumir premisas $\varphi_1\ldots\varphi_{n}$ cuya consequencia es $\varphi$}
            \item{Ahora asumir que $\varphi$ es falso $\neg\varphi$}
            \item{Encontrar una expressi\'on $\psi$ que pueda demostrarse cierta y falsa.}
        \end{enumerate} & 
        \[
        \begin{array}{l}
            \varphi_{1}\ldots\varphi_{n}\ \neg\varphi\ \psi \\
            \varphi_{1}\ldots\varphi_{n}\ \neg\varphi\ \neg\psi \\
            \hline
            \varphi_{1}\ldots\varphi_{n}\ \varphi \\
        \end{array}
        \]
        \\
    \end{tabular}
\end{frame}

\begin{frame}
    \frametitle{Abreviaci\'on}
    Es possible abreviar una secuencia (posiblemente vacia) de suposiciones
    como $\Gamma$.\\
    \[
        \begin{array}{l}
            \Gamma\ \neg\varphi\ \psi \\
            \Gamma\ \neg\varphi\ \neg\psi \\
            \hline
            \Gamma\ \varphi \\
        \end{array}
        \ \ \Leftrightarrow\ \ 
        \begin{array}{l}
            \varphi_{1}\ldots\varphi_{n}\ \neg\varphi\ \psi \\
            \varphi_{1}\ldots\varphi_{n}\ \neg\varphi\ \neg\psi \\
            \hline
            \varphi_{1}\ldots\varphi_{n}\ \varphi \\
        \end{array}
    \]
\end{frame}


\begin{frame}
    \frametitle{Regla del Antecedente}
    \[
        \begin{array}{l}
            \Gamma\ \ \varphi \\
            \hline
            \Gamma'\ \varphi \\
        \end{array}
        \ \ \mathtt{si}\ \  \Gamma \subset \Gamma'
    \]
    \begin{itemize}
        \item{Dice que si asumimos m\'as cosas, no podemos probar menos cosas}
    \end{itemize}
\end{frame}

\begin{frame}
    \frametitle{Regla de la premisa}
    \[
        \begin{array}{l}
             \\
            \hline
            \Gamma\ \varphi
        \end{array}
        \ \ \mathtt{if}\ \ \varphi\in\Gamma    
    \]
    \begin{itemize}
        \item{Simplemnte dice que si asumimos $\varphi$, podemos probar $\varphi$}
    \end{itemize}
\end{frame}

\begin{frame}
    \frametitle{Separaci\'on de casos}
    \[
        \begin{array}{l c l}
            \Gamma & \psi & \varphi \\
            \Gamma & \neg\psi & \varphi \\
            \hline
            \Gamma\ & & \varphi
        \end{array}
    \]
    \begin{itemize}
        \item{La expresi\'on $\varphi$ es consequencia de $\Gamma$ existe una
        formula $\psi$ tal que $\varphi$ es consequencia de $\Gamma\cup\psi$ y $\Gamma\cup\neg\psi$}
    \end{itemize}
\end{frame}

\begin{frame}
    \frametitle{Contradicci\'on}
    \[
        \begin{array}{l l}
            \Gamma & \psi \\
            \Gamma & \neg\psi \\
            \hline
            \Gamma & \varphi \\
        \end{array}
        \ \ \ \forall \varphi
    \]
    \begin{itemize}
        \item{El conjunto de premisas es inconsistente ya pue se puede
            demostrar $\psi$ y $\neg\psi$ con el.}
        \item{A partir de un conjunto de premisas inconsistente, se puede
            demostrar cualquier cosa.}
    \end{itemize}
\end{frame}

\begin{frame}
    \frametitle{Introducci\'on de $\vee$ en el antecedente}
    \[
        \begin{array}{l l l}
            \Gamma & \varphi & \xi \\
            \Gamma & \psi & \xi \\
            \hline
            \Gamma & \varphi \vee \psi & \xi
        \end{array}
    \]
    \begin{itemize}
        \item{Esta regla dice que si podemos demostrar $\xi$ utilizando
            $\Gamma$ y $\varphi$ o utilizando $\Gamma$ y $\psi$, tambien lo
            podemos demostrar utilizando $\Gamma$ y $\varphi\vee\psi$}
    \end{itemize}
\end{frame}

\begin{frame}
    \frametitle{Introducci\'on de $\vee$ en la conclusi\'on}
    \[
        \begin{array}{l l}
            \Gamma & \varphi \\
            \hline
            \Gamma & \psi \vee \varphi
        \end{array}
    \]
    \begin{itemize}
        \item{Si podemos probar $\psi$, tambien podemos probar $\psi\vee\varphi$}
    \end{itemize}
\end{frame}

\begin{frame}
    \frametitle{Introducci\'on de $\exists$ en la conclusi\'on}
    \[
    \begin{array}{l l}
        \Gamma & \varphi\frac{t}{x} \\
        & \\
        \hline
        & \\
        \Gamma & \exists x\varphi
    \end{array}
    \]
    \begin{itemize}
        \item{Dadas las premisas $\Gamma$}
        \item{Si encontramos un termino, el cula resulta cierto al
            substituirlo por la variable libre $x$ en una expression $\varphi$,
            demostramos la existencia de dicho $x$.}
    \end{itemize}
\end{frame}

\begin{frame}
    \frametitle{Introducci\'on de $\exists$ en las premisas}
    \[
        \begin{array}{l l l}
            \Gamma & \varphi\frac{t}{x} & \psi \\
            & & \\
            \hline
            & & \\
            \Gamma & \exists x\varphi & \psi
        \end{array}
    \]
    \begin{itemize}
        \item{Si asumimos $\Gamma$}
        \item{La expressi\'on $\psi$ es verdadera al reemplazar la variable
        $x$ con un ejemplo particular $t$ en la expressi\'on $\varphi$.}
        \item{Se puede concluir que $\psi$ tambien es verdadero con la
        expressi\'on $\exists x\varphi$}
    \end{itemize}
\end{frame}

\begin{frame}
    \frametitle{Reflexividad de ($=$)}
    \[
        \begin{array}{l}
            \hline
            t=t
        \end{array}    
    \]
    \begin{itemize}
        \item{Todo valor es igual a si mismo.}
    \end{itemize}
\end{frame}

\begin{frame}
    \frametitle{Substituci\'on por iguladad}
    \[
        \begin{array}{l l l}
            \Gamma & & \varphi \\
            \hline
            \Gamma & x=t & \varphi\frac{t}{x}
        \end{array}    
    \]
    \begin{itemize}
        \item{Si un termino se substituye por un termino equivalente,
            la expresi\'on no pierde su validez}
    \end{itemize}
\end{frame}

\begin{frame}
    \frametitle{Observaciones}
    \begin{itemize}
        \item{Esta es la lista completa del calculo para derivar expressiones.}
        \item{Utilizando estas reglas, se pueden derivar reglas nuevas.}
        \item{A pesar que las expressiones hablan sobre objetos reales y 
            semantica, estas derivaciones son puramente sintacticas.}
        \item{Una computadora (con suficiente tiempo), podria aplicar
            estas reglas en toda permutaci\'on possible y eventualmente
            probar o rechazar cualquier expressi\'on de logica de primer
            orden.}
        \item{A partir de la manipulaci\'on de simbolos, podemos enconrar
            conocimiento del mundo real.}
    \end{itemize}
\end{frame}

\begin{frame}
    \frametitle{Ejemplo}
    Derivaci\'on de la regla \emph{cadena}:
    \[
        \begin{array}{l l l}
            \Gamma & & \psi \\
            \Gamma & \varphi & \psi \\
            \hline
            \Gamma & & \psi \\
        \end{array}
    \]
\end{frame}

\begin{frame}
    \frametitle{}
\end{frame}

\begin{frame}
    \frametitle{Referencias}
    \bibliography{../Referencias/referencias}
    \bibliographystyle{plain}
\end{frame}

\end{document}